\PassOptionsToPackage{unicode=true}{hyperref} % options for packages loaded elsewhere
\PassOptionsToPackage{hyphens}{url}
%
\documentclass[]{book}
\usepackage{lmodern}
\usepackage{amssymb,amsmath}
\usepackage{ifxetex,ifluatex}
\usepackage{fixltx2e} % provides \textsubscript
\ifnum 0\ifxetex 1\fi\ifluatex 1\fi=0 % if pdftex
  \usepackage[T1]{fontenc}
  \usepackage[utf8]{inputenc}
  \usepackage{textcomp} % provides euro and other symbols
\else % if luatex or xelatex
  \usepackage{unicode-math}
  \defaultfontfeatures{Ligatures=TeX,Scale=MatchLowercase}
\fi
% use upquote if available, for straight quotes in verbatim environments
\IfFileExists{upquote.sty}{\usepackage{upquote}}{}
% use microtype if available
\IfFileExists{microtype.sty}{%
\usepackage[]{microtype}
\UseMicrotypeSet[protrusion]{basicmath} % disable protrusion for tt fonts
}{}
\IfFileExists{parskip.sty}{%
\usepackage{parskip}
}{% else
\setlength{\parindent}{0pt}
\setlength{\parskip}{6pt plus 2pt minus 1pt}
}
\usepackage{hyperref}
\hypersetup{
            pdftitle={Foundations of data science},
            pdfauthor={Steffen Mæland, PierGianLuca Porta Mana},
            pdfborder={0 0 0},
            breaklinks=true}
\urlstyle{same}  % don't use monospace font for urls
\usepackage{longtable,booktabs}
% Fix footnotes in tables (requires footnote package)
\IfFileExists{footnote.sty}{\usepackage{footnote}\makesavenoteenv{longtable}}{}
\usepackage{graphicx,grffile}
\makeatletter
\def\maxwidth{\ifdim\Gin@nat@width>\linewidth\linewidth\else\Gin@nat@width\fi}
\def\maxheight{\ifdim\Gin@nat@height>\textheight\textheight\else\Gin@nat@height\fi}
\makeatother
% Scale images if necessary, so that they will not overflow the page
% margins by default, and it is still possible to overwrite the defaults
% using explicit options in \includegraphics[width, height, ...]{}
\setkeys{Gin}{width=\maxwidth,height=\maxheight,keepaspectratio}
\setlength{\emergencystretch}{3em}  % prevent overfull lines
\providecommand{\tightlist}{%
  \setlength{\itemsep}{0pt}\setlength{\parskip}{0pt}}
\setcounter{secnumdepth}{5}
% Redefines (sub)paragraphs to behave more like sections
\ifx\paragraph\undefined\else
\let\oldparagraph\paragraph
\renewcommand{\paragraph}[1]{\oldparagraph{#1}\mbox{}}
\fi
\ifx\subparagraph\undefined\else
\let\oldsubparagraph\subparagraph
\renewcommand{\subparagraph}[1]{\oldsubparagraph{#1}\mbox{}}
\fi

% set default figure placement to htbp
\makeatletter
\def\fps@figure{htbp}
\makeatother

\usepackage{booktabs}
\usepackage{amsthm}
\makeatletter
\def\thm@space@setup{%
  \thm@preskip=8pt plus 2pt minus 4pt
  \thm@postskip=\thm@preskip
}
\makeatother
\usepackage[]{natbib}
\bibliographystyle{apalike}

\title{Foundations of data science}
\author{Steffen Mæland, PierGianLuca Porta Mana}
\date{2023-04-27}

\begin{document}
\maketitle

{
\setcounter{tocdepth}{1}
\tableofcontents
}
\hypertarget{preface}{%
\chapter*{Preface}\label{preface}}
\addcontentsline{toc}{chapter}{Preface}

Under construction

\hypertarget{intro}{%
\chapter{Introduction}\label{intro}}

\hypertarget{truth-inference-and-probability-inference}{%
\chapter{Truth inference and probability inference}\label{truth-inference-and-probability-inference}}

\hypertarget{statements-well-posed-and-ill-posed}{%
\section{Statements, well-posed and ill-posed}\label{statements-well-posed-and-ill-posed}}

Facts, hypotheses, questions, decisions -- and data are communicated through language and sentences. You may say ``well, data can be just numbers, they don't need to be communicated through sentences''. But is that true?

I give you this number: ``5''. OK it's a number, but what's it about? what should you do with it? is that ``data''? Instead, if I tell you: ``The number of lectures in this course is 5'' then I have given you a piece of information, a datum (even if it is actually false). Underlying any piece of information, hypothesis, or datum, there is always a statement that gives you the meaning and context of that datum.

In fact we face problems when those statements aren't clear. Suppose that
an electric-car model \href{https://ev-database.org/cheatsheet/energy-consumption-electric-car}{consumes
150~Wh/km}
and \href{https://ev-database.org/cheatsheet/range-electric-car}{has a range of
200~km}; a second
car model consumes 250~Wh/km and has a range of 600~km. Someone asks you:
``which model is better?''. Well, it isn't clear how you should answer; what
does ``better'' mean? If it refers to consumption, then the first car is
``better''. If it refers to range, then the second car is ``better''. If
it refers to a combination of these two characteristics, or to something
else, then you simply can't answer. Here we have a problem with querying and
giving data, because the statement underlying such query is not clear. We
say that statement is not \textbf{well-posed}, or that it is \textbf{ill-posed}.

This may seem an obvious discussion to you. Yet you'd be surprised by how often unclear statements appear in scientific papers about data engineering! Not seldom we find discussions and disagreements that actually come from unclear underlying statements, that two parties interpret in different ways.

As a data engineer, you'll often have the upper hand if you are on the lookout for ill-posed statements. Whenever you face an important question, or you're given an important piece of information, or you must provide an important piece of information, always take a little time to examine whether the question or information is actually well-posed.

\begin{itemize}
\tightlist
\item
  \emph{Exercise: give actual paper to analyse}
\end{itemize}

\hypertarget{literature}{%
\chapter{Literature}\label{literature}}

Here is a review of existing methods.

\bibliography{portamanabib.bib,book.bib,packages.bib}

\end{document}
